\documentclass[class=article]{standalone}

\begin{document}
	\section{Introduzione}
%	\begin{enumerate}
%		\item cosa e perchè si sta facendo quello che si sta facendo
%	\end{enumerate}

	Con la crescente attenzione verso l'impatto ambientale dei trasporti la bicicletta è destinata ad avere una rilevanza sempre maggiore.

	Il lavoro compiuto durante questa tesi è volto alla definizione di indicatori che consentano di caratterizzare lo stile di guida di un ciclista a partire dai dati raccolti da un sensore montato sopra a una bicicletta. 
%	Nel documento seguente verrà descritto come sono stati ottenuti i dati e come, successivamente, siano stati selezionati ed elaborati al fine di produrre gli indicatori stessi. 
	
%	rivedere
		
 	La caratterizzazione dello stile di guida non solo può aiutare i ciclisti a migliorare le loro prestazioni, ma può anche contribuire a promuovere una guida più sicura e a ottimizzare gli allenamenti in base alle esigenze individuali. 

	Nel corso di questa tesi, verranno approfonditi i dettagli di ciascuna fase, illustrando le metodologie impiegate, gli strumenti utilizzati e i risultati ottenuti. Infine, saranno discusse le implicazioni pratiche di questi indicatori e le possibilità di applicazione futura nel campo del ciclismo.

\end{document}