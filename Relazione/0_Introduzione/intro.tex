\documentclass[class=article]{standalone}

\begin{document}
	\section{Introduzione}	
	Con la crescente attenzione verso l’impatto ambientale dei trasporti, la bicicletta è destinata ad avere una rilevanza sempre maggiore. 
	
	Il lavoro compiuto durante questa tesi è volto alla definizione di indicatori che consentano di caratterizzare lo stile di guida di un ciclista a partire dai dati raccolti da un sensore montato sopra una bicicletta. La caratterizzazione dello stile di guida non solo può aiutare i ciclisti a migliorare le loro prestazioni, ma può anche contribuire a promuovere una guida più sicura e a ottimizzare gli allenamenti in base alle esigenze individuali.
	
	Nel corso di questa tesi, verranno approfonditi i dettagli di ciascuna fase, illustrando le metodologie impiegate, gli strumenti utilizzati e i risultati ottenuti. Infine, saranno discusse le implicazioni pratiche di questi indicatori e le possibilità di applicazione futura nel campo del ciclismo.\hfill\break
	
	La tesi si struttura nel modo seguente: nel capitolo 2 verrà fornita una breve panoramica  degli studi esistenti riguardanti la definizione di indicatori per la caratterizzazione dello stile di guida.
	
	Durante il capitolo 3 verrà fornita una base teorica fondamentale per comprendere i dati raccolti, analizzando i principali fenomeni fisici che influenzano la bicicletta durante il moto.
	
	Il capitolo 4 descriverà il sensore utilizzato e il sistema in generale, mentre gli esperimenti condotti e i dati raccolti verranno discussi durante il capitolo 5.
	
	Il capito 6, invece, si concentrerà sulla definizione degli indicatori. In particolare verranno approfonditi quelli relativi ad accelerazione, frenata e curva.
	
	Nel capitolo 7 verranno trattate le potenziali direzioni future della ricerca, discutendo di ulteriori rilievi e fornendo esempi di utilizzo degli indicatori.
	
	Nell'ultimo capitolo, infine, verranno presentate potenziali applicazioni dei risultati ottenuti nello sport e nell'ambito della sicurezza stradale.
\end{document}