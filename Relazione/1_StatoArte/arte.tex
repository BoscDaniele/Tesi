\documentclass[class=article]{standalone}

%\documentclass[12pt,a4paper]{article}
%
%\usepackage{../preambolo}
%
%\usepackage[
%style=numeric,
%natbib=true,
%backend=biber,
%sorting=none,
%maxbibnames=99
%]{biblatex}
%\addbibresource{../bibliografia.bib}

\begin{document}
%	\justifying
	\section{Stato dell'Arte}
	Sono già stati effettuati numerosi studi riguardo la definizione di indicatori per la caratterizzazione dello stile di guida.
	Mentre la maggior parte di questi si concentrano sulle automobili, altri, come in questo caso, si focalizzano su veicoli più leggeri, come le biciclette.
	
	Gli approcci utilizzati per la caratterizzazione dello stile di guida variano notevolmente sia per finalità che per realizzazione. Nel nostro caso, abbiamo scelto di utilizzare una piattaforma inerziale montata sul telaio della bicicletta. Tuttavia non è sempre questo il caso.
	
	Per esempio, lo studio \cite{ruota}, utilizza lo stesso sensore ma agganciato alla ruota posteriore. Questo gli consente di misurare direttamente la velocità della bicicletta e la cadenza della pedalata utilizzando queste informazioni al fine di monitorare le prestazioni di un ciclista.
	
	Lo studio \cite{brake}, invece, si concentra sul riconoscere un evento specifico, utilizzando una piattaforma inerziale per rilevare le frenate, distinguendo quelle lunghe e progressive da quelle improvvise.
	
	Vista la facile reperibilità, in altri articoli si è preferito utilizzare come sensore uno smartphone, i quali contengono al loro interno accelerometri, giroscopi e GPS. Questo approccio ha il vantaggio di sfruttare dispositivi già ampiamente diffusi.
	
	Tra questi citiamo lo studio \cite{transfun} che utilizza uno smartphone montato sul manubrio della bicicletta. Questo consente di registrare i movimenti periodici necessari per il bilanciamento della bici. I dati raccolti vengono poi analizzati tramite l'algoritmo di machine learning Random Forest Classifier per determinare cosa stia avvenendo.
	
	Infine, l'articolo \cite{dtwAndCo} propone tre diversi algoritmi per rilevare lo stile di guida di un veicolo, utilizzando uno smartphone come piattaforma inerziale e GPS. Grazie a quest'ultimo è possibile ottenere delle misure di velocità più precise rispetto all'integrazione dell'accelerazione. Il riconoscimento degli eventi avviene poi tramite introduzione di livelli soglia o tramite Dynamic Time Warping.
	
%	\newpage
%	\printbibliography[title={Bibliografia}]
	
\end{document}