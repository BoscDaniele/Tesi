\documentclass[class=article]{standalone}

\begin{document}
	\section{Stato dell'Arte}
	Sono già stati effettuati numerosi studi riguardanti la definizione di indicatori per la caratterizzazione dello stile di guida.
	Molti di questi si concentrano sulle automobili e altri, come in questo caso, su veicoli più leggeri, come le biciclette.
	
	Ogni ricerca aveva la sua peculiarità, nonostante l'argomento fosse simile, alcune di queste utilizzano il machine learning o il dynamic time warping per ottenere pattern utili per il riconoscimento, altri studi utilizzano metodi di analisi statistica e altri algoritmi per evidenziare eventi specifici.
	
	Le ricerche si differenziano anche in base ai sensori utilizzati, la maggior parte di queste fanno utilizzo di piattaforme inerziali oppure cellulari per raccogliere dati quali accelerazione e velocità angolari e ottengono poi la velocità e l'orientamento integrando le precedenti.
	
	In alcuni casi al fine di ottenere misure più attendibili la piattaforma inerziale è stata accompagnata da un altro sensore in grado di "catturare" la velocità del veicolo. In questi casi di solito si è optato per un GPS oppure per un altro sensore posto sulla ruota che consente di determinare in quanto tempo questa compie un giro completo e, quindi, la velocità della bicicletta.
	
	Altri articoli, per meglio catturare alcuni comportamenti del ciclista, hanno provato a montare il sensore sul manubrio della bicicletta invece che sulla canna. In questo modo il sensore è riuscito a catturare il movimento che il ciclista fa con il manubrio per mantenere in equilibrio la bicicletta durante la corsa e, in base a questo, stabilire che stile di guida stesse adottando.
	
	Infine altri articoli hanno fatto uso di telecamere e sensori montati sul ciclista stesso per valutare il comportamento di quest'ultimo durante la corsa.\hfill\break
	
	
	(Per la bicicletta elettrica è stato provato a valutare lo stile di guida a partire dalla tensione in entrata e in uscita dalla batteria)
	
	(In altri casi i dati raccolti venivano usati per prevedere la traiettoria della bicicletta o per capire quando il ciclista sta frenando)
	
	
%	Ogni ricerca aveva la sua peculiarità, nonostante l'argomento fosse simile, alcuni studi si concentravano sulla definizione di stili di guida mentre altri utilizzavano i dati raccolti per fini differenti come per esempio \Huge(Prendere qualche articolo già letto e inserirlo).
	
	
	
\end{document}