\documentclass[class=article]{standalone}

%\documentclass[12pt, a4paper]{article}
%
%\usepackage{../preambolo}

%\usepackage[
%style=numeric,
%natbib=true,
%backend=biber,
%sorting=none,
%maxbibnames=99
%]{biblatex}
%\addbibresource{../bibliografia.bib}

\begin{document}
%	\justifying	
	\section{Sviluppi Futuri}
	In questo capitolo andremo a trattare degli ulteriori rilievi che possono essere fatti al fine di migliorare la comprensione di ciò che sta accadendo. Inoltre faremo alcuni esempi di come è possibile interpretare i dati ottenuti dagli indicatori.
	
	\subsection{Ulteriori Rilievi}
	Per quanto riguarda la guida in piano, potrebbe essere interessante effettuare dei test \\relativi al cambio dei rapporti in corsa e analizzare come questo si riflette sugli indicatori. Probabilmente, l'effetto sarebbe visibile sulla media dell'accelerazione lungo l'asse \(x\) come una rampa, dovuta a una maggiore efficacia nel trasferimento della forza dai \\pedali alla ruota. Questo effetto si rifletterebbe anche sulla velocità e sarebbe, quindi, visibile anche nella varianza della stessa come un leggero picco, facilmente distinguibile dalle frenate, evento per cui la varianza è stata scelta come indicatore.\hfill\break
	
	Un altro aspetto interessante da analizzare è l'effetto delle salite e delle discese sui dati raccolti. Questi cambi di pendenza infatti, variando l'angolo di \(beccheggio\), influenzano tutte le misure prese. Affrontando una salita o una discesa si osserverebbe una variazione nella velocità angolare di questo angolo in funzione della velocità con cui ci avviciniamo alla pendenza e all'inclinazione stessa.
	
	Inoltre, la rotazione attorno all'angolo di \(beccheggio\) è visibile nelle misure di accelerazione. Come discusso nella sezione relativa alla fisica, il sensore percepisce la gravità come un vettore orientato verso l'alto. Di conseguenza, durante una discesa, si rileva un'accelerazione costante diretta dietro al ciclista, mentre durante una salita, l'accelerazione sarà diretta in avanti.\hfill\break
	
	Il problema dell'effetto della pendenza sulle misure di accelerazione può essere affrontato in due modi.
	
	Si può rimuovere il vettore gravità dalle misure ma, durante le discese, la bicicletta sembrerebbe ferma in quanto, in questi tratti, la componente principale del movimento è proprio la gravità.
	
	In alternativa si può cercare di capire quando ci stiamo muovendo su di un piano inclinato e aggiustare i dati di conseguenza. A tal fine si può utilizzare, per esempio, un sensore di pressione. Se la pressione aumenta, significa che ci si sta muovendo in discesa; se diminuisce, si sta affrontando una salita.
	
	
	\subsection{Utilizzo degli Indicatori}
	Gli indicatori precedentemente individuati possono essere utilizzati in diversi modi.
	
	Un modo semplice è l'introduzione di valori soglia. Quando gli indicatori superano un determinato valore, si può dedurre che un certo evento si sia verificato. Per esempio, quando la media della velocità angolare supera una certa soglia, si sa che la bicicletta sta curvando; oppure, quando la varianza della velocità supera un certo livello, si può dedurre che la bicicletta sta frenando.
	
	Questo approccio è stato utilizzato nell'articolo \cite{MA2021106096} al fine di dividere i rilievi fatti in categorie. Le categorie così ottenute sono state poi impiegate per addestrare un algoritmo di machine learning affinché potesse riconoscere gli eventi catalogati.
	
	In letteratura, gli algoritmi comunemente utilizzati per riconoscere se delle misure appartengono a un determinato evento sono il K-Nearest Neighbors \cite{transfun} e il Dynamic Time Warping \cite{dtw,dtwAndCo}.
	
\end{document}