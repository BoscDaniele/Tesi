\documentclass[class=article]{standalone}

%\documentclass[12pt, a4paper]{article}
%
%\usepackage{../preambolo}
%
%\usepackage[
%style=numeric,
%natbib=true,
%backend=biber,
%sorting=none,
%maxbibnames=99
%]{biblatex}
%\addbibresource{../bibliografia.bib}

\begin{document}
	\section{Conclusioni}
%	\justifying
%	In questa tesi abbiamo affrontato il tema dell'individuazione di indicatori per la caratterizzazione dello stile di guida dei ciclisti utilizzando sensori montati su una bicicletta.
	In questa tesi abbiamo affrontato il tema dell'individuazione di indicatori per la carat-\\terizzazione dello stile di guida di una bicicletta. Questo tipo di studio può avere diverse applicazioni, in particolare nell'ambito della sicurezza e dell'attività sportiva.\hfill\break
	
	Grazie agli indicatori individuati è possibile stabilire l'aggressività di un ciclista e la sua abitudine ad adottare comportamenti pericolosi come zig-zagare, impennare e, grazie all'integrazione di ulteriori sensori come una videocamera, andare contromano.
	
	Introducendo un sensore in grado di rilevare efficacemente la velocità della bicicletta, per esempio un GPS o un sensore magnetico agganciato alla ruota, è possibile calcolare la traiettoria della bicicletta.
	
	Monitorando l'attività e lo stile di guida di un atleta è possibile migliorare le sue prestazioni. Integrando il sistema con sensori biometrici è possibile raccogliere informazioni sullo stato di salute e affaticamento del ciclista, permettendo così di adattare l'allenamento in base alle esigenze.
	
	Un ulteriore campo di applicazione può essere il riconoscimento del pilota all'interno di un gruppo ristretto di persone, come un nucleo familiare, ricostruendo le abitudini e i pattern tipici nel modo di guidare del ciclista.
\end{document}