\documentclass[class=article]{standalone}

\begin{document}
	\section{Analisi dei Dati}
	\subsection{Raccolta Dati}	
	Nonostante siano state prese misure anche in salita e discesa, l'esperimento si è concentrato prevalentemente sull'identificazione di parametri su percorsi il più possibile piani. Questo perché manca un qualcosa che adesso non mi viene in mente come sia chiama per rimuovere la gravità dalle misure. Dal momento che il sensore percepisce la gravità come un'accelerazione che lo tira verso l'alto, evitando di farlo precipitare verso il centro della terra, e che la bicicletta quando affronta una salita o una discesa ruota attorno all'asse Y, abbiamo che durante il movimento in salita (discesa) il sensore ruota attorno all'asse Y in verso positivo (negativo), il vettore gravità si sposta quindi parzialmente sull'asse X facendo credere al sensore di stare accelerando (rallentando), falsando così le misure. Si è quindi preferito proseguire lo studio su superfici il più possibile piane.
	
	Ci sono stati tentativi di eliminare il problema della gravità utilizzando i filtri ahrsfilter e complementaryFilter che utilizzano le misure di accelerazione, velocità angolare e campo magnetico per ricavare, tramite un filtro di Kalman il primo e complementare il secondo, l'orientamento nel tempo della bicicletta. Con questo sarebbe stato possibile rimuovere la gravità applicandogli la matrice inversa di rotazione trovata. I due filtri però sono pensati per applicazioni dove il sensore accelera poco, infatti entrambi i filtri faticavano a stabilire quale fosse la gravità e quale fosse l'accelerazione della bicicletta.
	
	All'inizio di ogni sessione di acquisizione dati le prime misure sono state prese al fine di calcolare la matrice di rotazione per portare il sistema di riferimento dell'accelerometro a coincidere con quello della bicicletta. Questo è stato fatto mantenendo quest'ultima dapprima ferma in posizione verticale per ruotare la componente gravitazionale che giace sul piano XZ in modo da farlo coincidere con l'asse Z ruotando il sistema di riferimento attorno all'asse Y e, successivamente, inclinandola di lato in modo da portare una parte della gravità sul piano XY e ruotando quindi attorno all'asse Z per rendere nulla la componente lungo l'asse X.
	
	
	
	\subsection{Elaborazione Dati}
	Gli indicatori valutati sono quelli trovati nell'articolo \cite{chen} nella quale si prende in considerazione un'automobile.
	
	I dati raccolti durante i rilevamenti sono stati dapprima osservati così come sono, poi è stata eseguita l'analisi in frequenza, dell'accelerazione, per identificare dove si concentra la maggior parte dell'energia.
	Una volta identificato a che frequenze si trovano i disturbi il segnale è stato tagliato mediante un filtro passa-basso.
	
	Durante i rilievi, è stato utilizzato il sistema di riferimento NED (North-East-Down). Le direzioni positive degli assi sono quindi:
	\begin{itemize}
		\item di fronte alla bicicletta: la direzione positiva dell'asse X coincide con la direzione di movimento della bicicletta.
		\item a destra della bicicletta.
		\item sotto alla bicicletta
	\end{itemize}
	
	Dal momento che non viene eseguita la rimozione della gravità, questa appare come un vettore accelerazione di modulo pari all'accelerazione gravitazionale e direzionato verso l'alto, ovvero lungo l'asse negativo delle Z. Questo avviene perchè il sensore interpreta il fatto che non sta cadendo in caduta libera come una forza che lo tira verso l'alto evitandogli di cadere.
	
	Il magnetometro registra i dati del campo magnetico, in assenza di disturbi (qualsiasi cosa emetta un campo magnetico abbastanza potente da falsare le misure) può essere quindi utilizzato come una sorta di bussola. per questo motivo è molto comodo per individuare le curve.
	
	Al contrario delle automobili, le biciclette hanno la possibilità di rollare e impennare, questo si riflette sulla dinamica del veicolo e quindi sui dati raccolti.
	Ad esempio, durante una curva la bicicletta ruota attorno all'asse X e questo si riflette sulle letture dell'accelerometro, avremo infatti che, mentre l'accelerazione lungo il senso di marcia diminuisce, le accelerazioni lungo l'asse y e z variano a causa dello spostamento del vettore gravità che ora non si trova più sotto la bicicletta ma anche, parzialmente, a fianco.
	
\end{document}