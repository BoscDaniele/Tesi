\documentclass[class=article]{standalone}

\begin{document}
	\section{L'esperimento}
	Nonostante siano state prese misure anche in salita e discesa, l'esperimento si è concentrato prevalentemente sull'identificazione di parametri su percorsi il più possibile piani.
	
	All'inizio di ogni sessione di acquisizione dati le prime misure sono state prese al fine di calcolare la matrice di rotazione per portare il sistema di riferimento dell'accelerometro a coincidere con quello della bicicletta. Questo è stato fatto mantenendo quest'ultima dapprima ferma in posizione verticale per ruotare la componente gravitazionale che giace sul piano XZ in modo da farlo coincidere con l'asse Z ruotando il sistema di riferimento attorno all'asse Y e, successivamente, inclinandola di lato in modo da portare una parte della gravità sul piano XY e ruotando quindi attorno all'asse Z per rendere nulla la componente lungo l'asse X.
\end{document}