\documentclass[class=article]{standalone}

\begin{document}
	\section{Stato dell'Arte}
	Durante la lettura degli articoli per trovare qualcosa che fosse utile mi sono imbattuto in sistemi che volevano ottenere gli stessi risultati che volevo ottenere io ma adottando strategie differenti (O.O ??), utilizzando sensori aggiuntivi (normalmente GPS e/o telecamere), spostando la piattaforma inerziale sul manubrio al fine di rilevare il movimento dello stesso per mantenere la bicicletta in equilibrio (la rotazione periodica del manubrio può essere utilizzata come indicatore dello stile di guida e, per quanto visibile anche se il sensore è montato altrove), utilizzando più sensori inerziali dispiegati in diversi punti (della bicicletta e della persona), utilizzando anche dati biomedici (ecg,...), utilizzando la tensione erogata dalla batteria in caso di biciclette elettriche, utilizzando sensori per misurare la velocità della bicicletta o della ruota (magnete per vedere il tempo di rivoluzione della ruota, accelerometro montato sulla ruota).
\end{document}