\documentclass[12pt]{article}

\usepackage{preambolo}

\begin{document}
	% TITOLO
	\thispagestyle{empty}
\begin{titlepage}
\begin{center}
{{\Large{\textsc{Universit\`a degli studi di Bergamo}}}} \rule[0.1cm]{15.8cm}{0.1mm}
\rule[0.5cm]{15.8cm}{0.6mm}
{\small{\bf SCUOLA DI INGEGNERIA\\
Corso di Laurea Magistrale in Ingegneria Informatica}}
\end{center}

%%%%%%%%%%%%%%%%%%% UNIBG LOGO %%%%%%%%%%%%%%%%%%%%
\vspace{8mm}
\begin{figure}[h]
    \centering
    \includegraphics{unibg_logo.pdf}
\end{figure}
\vspace{8mm}
%%%%%%%%%%%%%%%%%%%%%%%%%%%%%%%%%%%%%%%%%%%%%%%%%%%

\begin{center}
{\LARGE{\bf Definizione di Indicatori per la Caratterizzazione dello Stile di Guida di Veicoli Leggeri\\}}
\vspace{5mm}
%{\Large{\bf Scaletta Elaborato Finale}}
\end{center}
\vspace{45mm}
\par
\noindent

\begin{center}
	{{\bf DANIELE BOSC}}
\end{center}
%\begin{minipage}[t]{0.55\textwidth}
%	\vfill
%\end{minipage}
%\hfill
%\begin{minipage}[t]{0.47\textwidth}\centered
%{{\bf DANIELE BOSC}}
%\end{minipage}
\vspace{15mm}
\begin{center}
{{\bf Anno Accademico 23/24}}
\end{center}    
\end{titlepage}
	
	% INDICE
	\pagenumbering{Roman}
	{
		\hypersetup{linkcolor=black}
		\renewcommand*\contentsname{Indice}
		\tableofcontents
	}
	
	\newpage
	\pagenumbering{arabic}
	\section{Introduzione}
	\begin{enumerate}
		\item cosa e perchè si sta facendo quello che si sta facendo
	\end{enumerate}
	
	\newpage
	\section{Stato dell'Arte}
	\begin{enumerate}
		\item come è stato affrontato il problema da altre parti
		\begin{enumerate}
			\item sensore sul manubrio
			\item encoder sulla ruota etc.
			\item (sensore di tensione in ingresso alla batteria di una bicicletta elettrica)
			\item ...
		\end{enumerate}
	\end{enumerate}
	
	\newpage
	\section{Fisica della Bicicletta}
	\begin{enumerate}
		\item breve descrizione di come funziona fisicamente una bicicletta (forze a cui è soggetta durante la corsa)
		\item cosa mi aspetto di vedere dai dati raccolti dal sensore
		\begin{enumerate}
			\item percorso rettilineo
			\item curva
			\item salita/discesa
		\end{enumerate}
	\end{enumerate}
	
	\newpage
	\section{Il Sistema}
	\begin{enumerate}
		\item da che elementi è composto il sistema utilizzato per raccogliere i dati
		\begin{enumerate}
			\item bicicletta
			\item sensore
		\end{enumerate}
	\end{enumerate}
	
	\newpage
	\section{Raccolta Dati}
	\begin{enumerate}
		\item come e dove si sono svolti gli "esperimenti" di raccolta dati
		\begin{enumerate}
			\item Orientamento del Vettore Gravitazionale
			\item Percorso Rettilineo
			\item Curva
			\item Curva a U
		\end{enumerate}		
		\item dati raccolti dal sistema
		\begin{enumerate}
			\item Accelerazione
			\item Velocità Angolare
			\item Campo Magnetico
			\item Velocità
		\end{enumerate}
		durante le fasi di
		\begin{enumerate}
			\item Accelerazione/Decelerazione
			\item Curva
			\item Frenata
		\end{enumerate}
	\end{enumerate}
	
	\newpage
	\section{Indicatori}
	\begin{enumerate}
		\item Indicatori presi in considerazione
		\begin{enumerate}
			\item media, media rettificata
			\item root mean square (rms)
			\item varianza, deviazione standard
			\item massimo, minimo e distanza picco-picco
			\item etc
		\end{enumerate}
		\item come si comportano gli indicatori durante le fasi di
		\begin{enumerate}
			\item Accelerazione/Decelerazione
			\item Curva
			\item Frenata
		\end{enumerate}
		e quali scegliere al fine di stabilire cosa e come (con che "intensità") sta avvenendo.
	\end{enumerate}
	
	\newpage
	\section{Sviluppi Futuri}
	\begin{enumerate}
		\item ulteriori test
		\begin{enumerate}
			\item Cambio dei Rapporti (Marcia)
			\item Salita/Discesa
		\end{enumerate}
		\item come interpretare i dati ottenuti
		\begin{enumerate}
			\item impostare di valori soglia
			\item machine learning
			\item dynamic time warping
			\item altro
		\end{enumerate}
		
	\end{enumerate}
	
	\newpage
	\section{Conclusioni}
	\begin{enumerate}
		\item come utilizzare gli indicatori ottenuti
		\begin{enumerate}
			\item capire quanto un forte sta andando un ciclista
			\item sicurezza/riconoscimento di comportamenti pericolosi (zig-zag, etc)
			\item riconoscimento di chi sta guidando all'interno di un gruppo di persone (per esempio un nucleo familiare)
		\end{enumerate}
	\end{enumerate}
	
\end{document}